\subsection{A Full Protocol Description}
\label{subsec:protocol-desc}

We introduce the main \protocFairLedger protocol instance that dispatches to the relevant subprocesses.
%
This protocol is parameterized with the \CPLen the common prefix, $R$ the recency parameter, $m$ the public-key subset size and \PBWindowLen the profile window length.
%
Refer to~\cref{table:main-parameters-protocol} in~\cref{sec:glossary} for a detailed explanation.

\begin{cccProtocol}
    {\protocFairLedger}
    {main-protocol}
    {The main protocol instance of \protocFairLedger.}
    
    \paragraph{Global Variables:}
    %
    \begin{cccItemize}[nosep]
        \item Parameters: $k, R, m, \PBWindowLen$
        
        \item Local states: $\round, \chainLocal, \buffer_{\mathsf{tx}}, \buffer_\PB, \buffer_\pk$
    \end{cccItemize}

    \paragraph{Registration/Deregistration:}
    %
    \begin{cccItemize}[nosep]
        \item Upon receiving $(\textsc{register}, \mathcal{R})$, where $\mathcal{R} \in \{ \funcFairLedger, \funcClock, \funcGRO, \funcGrpoRO, \funcEnclave \}$, execute $\mathsf{Registration}(\party, \sid, \mathtt{Reg}, \mathcal{R})$.

        \item Upon receiving $(\textsc{de-register}, \mathcal{R})$, where $\mathcal{R} \in \{ \funcFairLedger, \funcClock, \funcGRO, \funcGrpoRO, \funcEnclave\}$, execute $\mathsf{Deregistration}(\party, \sid, \mathtt{Reg}, \mathcal{R})$.

        \item Upon receiving input $(\textsc{is-registered}, \sid)$ return $(\textsc{register}, \sid, 1)$ if the local registry \texttt{Reg} indicates that this party has successfully completed a registration with $\mathcal{R} = \funcFairLedger$ (and did not de-register since then).
        %
        Otherwise, return $(\textsc{register}, \sid, 0)$.
    \end{cccItemize}

    \paragraph{Interacting with the Ledger:}
    %
    Upon receiving a ledger-specific input $I \in \{(\textsc{submit}, \ldots),\allowbreak (\textsc{read}, \ldots), (\textsc{maintain-ledger}, \ldots) \}$ verify first that all resources are available.
    %
    If not all resources are available, then ignore the input; else (i.e., the party is operational and time-aware) execute one of the following steps depending on the input $I$:
    %
    \begin{cccItemize}[nosep]
        \item \textbf{If} $I = (\textsc{submit}, \sid, \tx)$ then invoke $\mathsf{IssueNewTransaction(\party, \sid, \tx)}$.

        \item \textbf{If} $I = (\textsc{maintain-ledger}, \sid, \mathrm{minerID})$ \textbf{then} invoke $\mathsf{LedgerMaintenance}(\chainLocal, \party)$; if \textsf{LedgerMaintenance} halts then halt the protocol execution (all future input is ignored).

        \item If $I = (\textsc{read}, \sid)$ then invoke $\mathsf{ReadState}(\party, \sid)$.
    \end{cccItemize}

    \paragraph{Handling external calls:}
    %
    \begin{cccItemize}[nosep]
        \item Upon receiving $(\textsc{clock-read}, \sid_C)$ forward it to \funcClock and output \funcClock's response.
        \item Upon receiving $(\textsc{clock-update}, \sid_C)$, record that a clock-update was received in the current round.
        %
        If this protocol instance is currently only registered to the clock (and no other functionality), then forward $(\textsc{clock-update}, \sid_C)$ to \funcClock.
    \end{cccItemize}
\end{cccProtocol}

\paragraph{Registration and de-registeration.}
%
In order to participate in the protocol, parties need to register with their resources.
%
\cref{protocol:registration} captures the registration procedure.
%
Note that before registering with the network and random oracle, parties should check if they have access to all global functionalities.

\begin{cccProtocol}
    {$\mathsf{Registration}(\party, \sid, \mathtt{Reg}, \mathcal{G})$}
    {registration}
    {The registration procedure of \protocFairLedger.}

    \begin{algorithmic}[1]
        \If{$\mathcal{G} \in \{\funcClock, \funcEnclave \}$}
        
        \State{Send $(\textsc{register}, \sid)$ to $\mathcal{G}$, set registration status to registered with $\mathcal{G}$, and output the valued received by $\mathcal{G}$.}
        
        \ElsIf{$\mathcal{G} = \funcFairLedger$}
        
        \If{the party is not registered with \funcClock, \funcGRO, \funcGRO or \funcEnclave or is already registered with all setup functionalities}
        \State{ignore this input}
        \Else
        \State{Send $(\textsc{register}, \sid)$ to \funcDiffuse.}
        \State{Output $(\textsc{register}, \sid, \party)$ once completing the registration with all the above resources.}
        \EndIf
        
        \EndIf
    \end{algorithmic}
\end{cccProtocol}

The de-registration procedure (\cref{protocol:deregistration}) is analogous to the above, where de-registering from the ledger is simplified as dropping from the network functionality.

\begin{cccProtocol}
    {$\mathsf{Deregistration}(\party, \sid, \mathtt{Reg}, \mathcal{G})$}
    {deregistration}
    {The registration procedure of \protocFairLedger.}

    \begin{algorithmic}[1]
        \If{$\mathcal{G} \in \{\funcClock, \funcGRO, \funcGrpoRO, \funcEnclave \}$}
        \State{Send $(\textsc{de-register}, \sid)$ to $\mathcal{G}$, set registration status as de-registered with $\mathcal{G}$, and output the valued received by $\mathcal{G}$.}
        \EndIf

        \If{$\mathcal{G} = \funcFairLedger$}
        \State{Send $(\textsc{de-register}, \sid)$ to \funcDiffuse, set its registration status as de-registered with \funcDiffuse and output $(\textsc{de-register}, \sid, \party)$.}
        \EndIf
    \end{algorithmic}
\end{cccProtocol}

\paragraph{The transactions issuing procedure.}
%
In order to issue a transaction \tx, party \party needs to encrypt \tx using a subset of on-chain public keys, computed from a ticket associated with a block in the settled blockchain.
%
Additionally, a NIZKPoK that proves the validity of \tx and the correctness of each ciphertext should be attached (See the relation in Equation~\eqref{eq:nizk-relation}).

\begin{cccProtocol}
    {$\mathsf{IssueNewTransaction}(\party, \sid, \tx)$}
    {issue-new-transaction}
    {The procedure to issue a new transaction in \protocFairLedger.}

    \begin{algorithmic}[1]
        \State{$h \gets \chainHead{\chainPrefix{\chainLocal}{\CPLen}}$ and $\st \gets$ blockchain state associated with $h$}
        \State{$\mathcal{S}_{\pk} \gets$ all enclave public keys up to block with hash $h$}

        \State{$\mathcal{S}^\ticket_{\pk} \gets f_{\mathsf{select}}(\mathcal{S}_{\pk}, m, \ticket)$ and $m' \gets |\mathcal{S}^\ticket_{\pk}|$}

        \For{$i$ \textbf{from} $1$ \textbf{to} $m'$}
        \State{$\pk_i \gets \mathcal{S}^\ticket_{\pk}[i], r_i \overset{\$}{\gets} \{0, 1\}^\kappa$}
        \State{$ct_i \gets \PKE.\Enc(\pk_i, \tx, h;r_i)$}
        \EndFor

        \LineComment{Generate NIZK proof.}

        \State{$\pi \gets \NIZK.\Prove((h, \st, (ct_1, \ldots, ct_{m'}), (\pk_1, \ldots, \pk_{m'})), (\tx,\allowbreak (r_1, \ldots, r_{m'})))$}

        \State{$\buffer_{\mathsf{tx}} \gets \buffer_{\mathsf{tx}} \concat (\pi, ct_1, \ldots, ct_{m'})$}

        \State{Send $(\textsf{diffuse}, \sid, (\pi, ct_1, \ldots, ct_{m'}, \ticket))$ to $\funcDiffuse^{\mathsf{tx}}$}
    \end{algorithmic}
\end{cccProtocol}

\paragraph{Fetch information.}
%
Parties fetch information from different diffusion functionalities --- $\funcDiffuse^{\mathsf{bc}}$, $\funcDiffuse^{\mathsf{tx}}$, $\funcDiffuse^{\mathsf{pb}}$, $\funcDiffuse^\pk$ --- to learn the new chains, profile blocks, transactions and enclave public keys, respectively.
%
When receiving an encrypted transaction, they verify if it provides a valid block ticket and whether it attaches a good NIZK proof.
%
If the transaction verifies, book-keep its local receiving time.

\begin{cccProtocol}
    {$\mathsf{FetchInformation}(\party, \sid)$}
    {fetch-information}
    {The fetch-information procedure in \protocFairLedger.}
    
    \begin{algorithmic}[1]
        \LineComment{Fetch blocks.}
        \State{Send $(\textsc{fetch}, \sid)$ to $\funcDiffuse^{\mathsf{bc}}$; denote the response from $\funcDiffuse^{\mathsf{bc}}$ by $(\textsc{fetch}, \sid, b)$.}
        \State{Extract chains $\chain_1, \ldots , \chain_k$ from $b$.}
        \State{$\chainLocal \gets \mathsf{maxvalid}(\chainLocal, \chain_1, \ldots, \chain_k)$}

        \LineComment{Fetch profile blocks.}
        \State{Send $(\textsc{fetch}, \sid)$ to $\funcDiffuse^{\mathsf{pb}}$; denote the response from $\funcDiffuse^{\mathsf{pb}}$ by $(\textsc{fetch}, \sid, b)$.}
        \State{Extract profile blocks $\PB_1, \ldots , \PB_k$ from $b$.}
        \State{Set $\buffer_\PB \gets \buffer_\PB \concat (\PB_1, \ldots, \PB_k)$}

        \LineComment{Fetch enclave public keys and tickets.}
        \State{Send $(\textsc{fetch}, \sid)$ to $\funcDiffuse^\pk$; denote the response from $\funcDiffuse^\pk$ by $(\textsc{fetch}, \sid, b)$.}
        \State{Extract all public keys $\mathsf{pk}_1, \ldots , \mathsf{pk}_k$ from $b$.}
        \State{Set $\buffer_\pk \gets \buffer_\pk \concat (\mathsf{pk}_1, \ldots, \mathsf{pk}_k)$}

        \LineComment{Fetch encrypted transactions.}
        \State{Send $(\textsc{fetch}, \sid)$ to $\funcDiffuse^{\mathsf{tx}}$; denote the response from $\funcDiffuse^{\mathsf{tx}}$ by $(\textsc{fetch}, \sid, b)$.}

        \State{Extract transactions $(\pi_1, ct_1, \ticket_1), \ldots , (\pi_k, ct_k, \ticket_k)$ from $b$.}

        \For{$i$ \textbf{from} $1$ \textbf{to} $k$}
        \If{$\ticket_i$ is valid \textbf{and} $\NIZK.\Verify(\pi) = 1$ \textbf{and} $h, st \in \pi$ match that on \chainLocal}
        \State{$\buffer_{\mathsf{tx}} \gets \buffer_{\mathsf{tx}} \concat ((\pi_i, ct_i, \ticket_i), \round)$} \Comment{Bookkeep local receiving time \round}
        \EndIf
        \EndFor
    \end{algorithmic}
\end{cccProtocol}

\paragraph{Chain validation rules.}
%
The validation procedure (\cref{algorithm:isvalidchain}) generally follows that in Bitcoin backbone protocol, plus additionally verifying the validity and freshness of profile blocks.

For simplicity, we use $\mathsf{validBlock}^T$ to verify if a block is a successful PoW.
%
\[ \mathsf{validBlock}^T(ctr, r, h, st, h', st') \defeq H(ctr, r, h, st, h', val) < T \wedge ctr < 2^{32} .\]
%
Similarly, we use use $\mathsf{validProfile}^T$ to verify if a profile block is a successful PoW by checking the reverse of the string.

\begin{cccAlgorithm}
    {$\mathsf{IsValidChain}(\chain)$}
    {isvalidchain}
    {The chain validation rule in \protocFairLedger.}

    \begin{algorithmic}[1]
        \oneLineIf{\chain starts with a block other than \textbf{G}}{\Return \false}
        \oneLineIf{\chain encodes an invalid state with $\mathsf{isvalidstate}(\vec{\st}) = 0$}{\Return \false}
        \oneLineIf{\chain contains profile blocks with invalid content}{\Return \false}

        \State{$r' \gets \round, \block \gets \chainHead{\chain}, h^* \gets H(\block)$}
        \While{$\chain \neq \varepsilon$}
        \State{$\mathsf{isValid} \gets \true$}

        \LineComment{Check validity of block header}
        \oneLineIf{$(\mathsf{validBlock}^T(\block) = \false) \vee (h^* \neq H(\block)) \vee (\timestamp{\block} \ge r')$}{$\mathsf{isValid} \gets \false$}

        \LineComment{Check validity of profile blocks in \block}
        \For{$\PB \in \block$}

        \State{Parse \PB as $\langle \cdot, r, \cdot, \cdot, h', \cdot \rangle$}
        \State{$\block' \gets $ the block in \chain s.t. $H(\block) = h'$}

        \LineComment{Check validity of profile header}
        \oneLineIf{$(\mathsf{validProfile}^T(\PB) = \false) \vee r \ge \timestamp{\block}$}{$\mathsf{isValid} \gets \false$}

        \oneLineIf{$(\block' = \varepsilon) \vee r \ge \timestamp{\block'} + R$}{$\mathsf{isValid} \gets \false$}
        \Comment{Check freshness and recency}

        \EndFor

        \If{$\mathsf{isValid} = \true$}
        \State{$r' \gets r, h^* \gets h$}
        \State{Remove the rightmost block in \chain}
        \State{$\block \gets \chainHead{\chain}$}
        \Else
        \State{\Return \false}
        \EndIf

        \EndWhile

        \State{\Return \true}
    \end{algorithmic}
\end{cccAlgorithm}

\paragraph{Longest chain selection.}
%
Parties use the same longest-chain rule (\cref{algorithm:maxvalid}) as the Bitcoin backbone protocol to select their working chain.

\begin{cccAlgorithm}
    {$\mathsf{maxvalid}(\chain_1, \ldots, \chain_k)$}
    {maxvalid}
    {The chain selection rule in \protocFairLedger.}

    \begin{algorithmic}[1]
        \State{$\chain_{\mathsf{max}} \gets \varepsilon$}
        
        \For{$i$ \textbf{from} $1$ \textbf{to} $k$}
        \oneLineIf{$\mathsf{IsValidChain}(\chain_i)$ \textbf{and} $\chainLength{\chain_i} > \chainLength{\chain_{\mathsf{max}}}$}{$\chain_{\mathsf{max}} \gets \chain_i$}
        \EndFor
        
        \State{\Return $\chain_{\mathsf{max}}$}
    \end{algorithmic}
\end{cccAlgorithm}

\paragraph{The mining procedure.}
%
Parties use \twoforone PoW to extend the blockchain and mine new profile blocks.
%
If they succeed on either procedure, they diffuse the extended chain and new profile blocks to the corresponding diffusion network.

\begin{cccProtocol}
    {$\mathsf{MiningProceudre}(\party, \sid)$}
    {mining-procedure}
    {The mining procedure in \protocFairLedger.}

    \begin{algorithmic}[1]
        \LineComment{The following steps are executed in an $(\textsc{maintain-ledger}, \sid, \mathrm{minerID})$-interruptible manner:}

        \State{Set $st \gets$ Merkle root of profile blocks in $\buffer_\PB$, decrypted transactions in $\buffer_{\mathsf{tx}}$ and public keys in $\buffer_\pk$ that are not mined in \chainLocal}

        \State{Set $st' \gets$ Merkle root of $((\pi_1, t_1, \ticket_1), \ldots, (\pi_k, t_k, \ticket_k))$ where $\pi_i$ is a transaction that is not settled in \chainLocal and $t_i$ its local receiving time}

        \State{$h \gets H(\chainHead{\chainLocal}), h' \gets H(\chainHead{\chainPrefix{\chainLocal}{\CPLen}})$}

        \State{$u \gets \mathsf{H}(ctr, \round, h, h', st, st')$}

        \If{$u < T$}
        \State{$\block \gets \langle ctr, h, h', \round, st, st' \rangle$ and $\chainLocal \gets \chainLocal \concat \block$}
        \State{Send $(\textsc{diffuse}, \sid, \chainLocal)$ to $\funcDiffuse^{\mathsf{bc}}$ and proceed from here upon next activation of this procedure}
        \EndIf

        \If{$\stringRev{u} < T$}
        \State{$\PB \gets \langle ctr, h, h', \round,  st, st' \rangle$}
        \State{Send $(\textsc{diffuse}, \sid, \PB)$ to $\funcDiffuse^{\mathsf{pb}}$ and proceed from here upon next activation of this procedure}
        \EndIf

        \State{$ctr \gets ctr + 1$}
    \end{algorithmic}
\end{cccProtocol}

\paragraph{Ledger Maintenance.}
%
We group all the steps in the main ledger operation in $\mathsf{LedgerMaintenance}$.

\begin{cccProtocol}
    {$\mathsf{LedgerMaintenance}(\party, \sid)$}
    {ledger-maintenance}
    {The ledger maintenance procedure in \protocFairLedger.}

    \begin{algorithmic}[1]
        \LineComment{The following steps are executed in an $(\textsc{maintain-ledger}, \sid, \mathrm{minerID})$-interruptible manner:}

        \State{Invoke $\mathsf{FetchInformation}(\sid, \party)$ to receive the newest messages for this round}

        \LineComment{Decrypt transactions}
        \For{$(\pi, ct, \ticket) \in \chainLocal$}
        \If{$\chainLength{\chainLocal} > \block+ \Lambda$ where \block is a block with hash $h$ in $\pi$ \textbf{and} $\mathsf{pk}_i \in \pi$ is miner's enclave public key}
        \State{Send $(\textsc{tx-dec}, ct_i, \chainLocal)$ to \funcEnclave and get response $(\tx, ct, \sigma)$}
        \State{Add $(\tx, ct, \sigma)$ to $\buffer_{\mathsf{tx}}$}
        \EndIf
        \EndFor

        \If{$t_{\mathsf{work}} < \tau$}
        \State{Call $\mathsf{MiningProcedure}(\party, \sid)$}
        \State{Set $t_{\mathsf{work}} \gets \tau$}
        \EndIf

        \While{A $(\textsc{clock-update}, \sid_C)$ has not been received during the current round}
        \State{Give up activation (set the anchor here)}
        \EndWhile

        \State{Send $(\textsc{clock-update}, \sid_C)$ to \funcClock.}
        \Comment{Party will lose its activation here}
    \end{algorithmic}
\end{cccProtocol}

\paragraph{Reading the state.}
%
Upon receiving a \textsc{read} command, parties extract a transaction list from their local chain \chainLocal.
%
Recall that transaction inclusion and consensus are decoupled, we first introduce an algorithm $\mathsf{ExtractTransactionSequence}$ such that, taking input a chain \chain, converts it to a sequence of blocks of transactions \txSeqProposal, which are extracted using the median timestamp.
%
Note that \txSeqProposal shares the same length as \chain.

\begin{cccAlgorithm}
    {$\mathsf{ExtractTransactionSequence}(\chain)$}
    {extract-transaction-sequence}
    {Extract transaction sequence from a chain.}

    \begin{algorithmic}[1]
        \State{Initialize $\mathsf{txList} \gets \emptyset, \txSeqProposal \gets \varepsilon$}

        \LineComment{Extract transactin timestamps}
        \For{$i$ \textbf{from} $1$ \textbf{to} $\chainLength{\chain} - \PBWindowLen$}

        \For{$\txTag = (ct, \pi, \ticket) \in \PB \in \block_i$}
        \State{Set $B = \{\block_j \mathbin | i \le j < i + \PBWindowLen\}$}
        \State{$t \gets \med(\{t \mathbin|  (\txTag, t) \in \PB \in B  \} \cup \{+\infty \mathbin|  \txTag \not\in \PB \in \block \})$}
        \oneLineIf{$t \neq + \infty$}{Add $(\txTag, t)$ to $\mathsf{txList}$}
        \EndFor
        \EndFor

        \LineComment{Construct transaction sequence}

        \For{$i$ \textbf{from} $1$ \textbf{to} $\chainLength{\chain}$}
        \State{Set $\txSeqProposal_i \gets \varepsilon$}

        \If{$i \ge \PBWindowLen + \CPLen$}
        \State{Let $\vec{\txTag}$ denote the set of transaction tag s.t. the \PBWindowLen-window of \txTag ends in $\block_{i - k}$}

        \State{Order $\vec{\txTag}$ non-decreasingly based on timestamp}

        \For{$\txTag \in \vec{\txTag}$}
        \If{there is $\txTag'$ in block $\block_{i - k + 1}, \ldots \block_{i}$ s.t. $\timestamp{\txTag'} < \timestamp{\txTag}$}
        \State{$\txSeqProposal_i \gets \txSeqProposal_i \concat \tilde{\txTag}_1 \concat \ldots \tilde{\txTag}_k \concat \txTag$ where $\tilde{\txTag}_1 \concat \ldots \tilde{\txTag}_k$ are \txTag that are in $\mathsf{txList}$ with $\timestamp{\tilde{\txTag}_i} < \timestamp{\txTag}$ however not in \txSeqProposal}
        \EndIf
        \EndFor
        \EndIf

        \State{$\txSeqProposal \gets \txSeqProposal \concat \txSeqProposal_i$}

        \EndFor

        \State{\Return \txSeqProposal}
    \end{algorithmic}
\end{cccAlgorithm}

When the decryption of a transaction \txTag is available, that tag is replaced with the plain transaction.
%
A state $\vec{\st}$ is acquired by extracting a prefix of the transaction sequence \txSeqProposal such that all transactions are decrypted.

\begin{cccProtocol}
    {$\mathsf{ReadState}(\party, \sid)$}
    {read-state}
    {The read state procedure in \protocFairLedger.}

    \begin{algorithmic}[1]
        \State{Invoke $\mathsf{FetchInformation}(\sid, \party)$ to receive the newest messages for this round}

        \State{$\txSeqProposal \gets \mathsf{ExtractTransactionSequence}(\chainLocal)$}

        \For{$\txTag = (\pi, ct, \ticket) \in \txSeqProposal$}
        \If{there is a $\tx \in \chainLocal$ such that \tx is a signed decryption for $ct$}
        \State{Replace \txTag with \tx}
        \EndIf
        \EndFor

        \State{Extract the state $\vec{\st}$ from $\blockify(\txSeqProposal')$, where $\txSeqProposal'$ is a prefix of \txSeqProposal up to a block such that all transactions are decrypted}

        \State{Output $(\textsc{read}, \sid, \vec{\st}^{\lceil \CPLen})$ to \Z.}
    \end{algorithmic}
\end{cccProtocol}
