\section{Preliminaries (Cont'd)}
\label{sec:preliminaries-contd}

\paragraph{Global clock.}
%
We model synchronous processors using \funcClock (cf.~\cite{TCC:KMTZ13}).
%
At a high level, \funcClock maintains a round variable $\tau$ for each session, and the round forwards only when all registered parties send \funcClock the \textsc{clock-update} command.
%
The current time can be checked by sending a \textsc{clock-read} command.
%
Whenever a party who has finished the computation in a round is activated, she checks if the time from \funcClock has been forwarded; if not, she does nothing and wait for the next activation.

\begin{cccFunctionality}
    {\funcClock}
    {clock}
    {The global clock.}

    This functionality maintains state variables as follows.

    \addtocounter{table}{-1}
    \begin{tabularx}{.9\textwidth}{c X}
        \toprule[.3mm]
        \textbf{State Variable}
         & \textbf{Description}
        \\ \midrule[.3mm]
        $\partyset \gets \emptyset$
         & The set of registered parties $\party = (\pid, \sid)$.
        \\ \midrule
        $F \gets \emptyset$
         & The set of registered functionalities (together with their session identifier).
        \\ \midrule
        $\tau_\sid \gets 0$
         & The clock variable for session \sid.
        \\ \midrule
        $d_\party \gets 0$
         & The clock-update variable for $\party = (\pid, \sid) \in \partyset$.
        \\ \midrule
        $d(\F, \sid) \gets 0$
         & The clock-update variable for $(\F, \sid) \in F$.
        \\ \bottomrule[.3mm]
    \end{tabularx}

    \begin{cccItemize}[noitemsep]
        \item Upon receiving $(\textsc{clock-update}, \sid_C)$ from some party $\party \in \partyset$ set $d_\party \gets 1$; execute \emph{Round-Update} and forward $(\textsc{clock-update}, \sid_C, \party)$ to \adv.

        \item Upon receiving $(\textsc{clock-update}, \sid_C)$ from some functionality \F in a session \sid such that $(\F, \sid) \in F$ set $d(\F, \sid) \gets 1$, execute \emph{Round-Update} and return $(\textsc{clock-update},\allowbreak \sid_C, \F)$ to this instance of \F.

        \item Upon receiving $(\textsc{clock-read}, \sid_C)$ from any participant (including the environment on behalf of a party, the adversary, or any ideal—shared or local—functionality) return $(\textsc{clock-read}, sid_C , \tau_{\sid})$ to the requestor (where \sid is the \sid of the calling instance).
    \end{cccItemize}

    \emph{Procedure Round-Update:} For each session \sid do: If $d(\F, \sid) = 1$ for all $\F \in F$ and $d_\party = 1$ for all honest partyset $P = (\cdot, \sid) \in \partyset$, then set $\tau_\sid \gets \tau_\sid + 1$ and reset $d(\F, \sid) \gets 0$ and $d_\party \gets 0$ for all partyset $\party = (\cdot, \sid) \in \partyset$.
\end{cccFunctionality}


\paragraph{Diffuse functionalities.}
%
We model the \delay-bounded delay network with \funcDiffuse \cite{C:BMTZ17}.
%
Note that once a corrupted message reaches at least one honest party at round $r$, \funcDiffuse guarantees to deliver this message to all honest parties before round $r + \delay$ (i.e., honest parties keep ``echoing'' messages).
%
By convention, different types of messages are diffused by different functionalities, and we write $\funcDiffuse^{\mathsf{bc}}$, $\funcDiffuse^{\mathsf{tx}}$, $\funcDiffuse^{\mathsf{pb}}$, $\funcDiffuse^\pk$ to denote the network for chains, transactions, profile blocks and enclave public keys respectively.

\begin{cccFunctionality}
      {\funcDiffuse}
      {diffuse}
      {The diffusion network.}

      \newcommand*{\msgid}{\ensuremath{\mathsf{mid}}\xspace}
      \newcommand*{\vecM}{\ensuremath{\vec{M}}\xspace}

      The functionality is parameterized by the network delay \delay.

      \begin{minipage}{\linewidth}
            \begin{tabularx}{.9\textwidth}{c  X}
                  \toprule[.3mm]
                  \textbf{State Variable}
                   & \textbf{Description}
                  \\ \midrule[.3mm]
                  $\partyset \gets \emptyset$
                   & The set of registered parties.
                  \\ \midrule
                  $\vecM \gets [] $
                   & A dynamically updatable list of quadruples $(m, \msgid, D_{\msgid}, \party)$ where $D_{\msgid}$ denotes the fetch counter.
                  \\ \bottomrule[.3mm]
            \end{tabularx}
            \addtocounter{table}{-1}
      \end{minipage}

      \paragraph{Network capabilities:}
      %
      \begin{cccItemize}[nosep]
            \item Upon receiving $(\textsc{diffuse}, \sid, m)$ from some
            $\party_s \in \partyset$, where $\partyset = \{ \party_1, \ldots, \party_n \}$ denotes the current party set, do:
            %
            \begin{cccEnum}[nosep]
                  \item Choose $n$ new unique message-IDs $\msgid_1, \ldots, \msgid_n$.
                  \item Initialize $2n$ new variables $D_{\msgid_1} := D^{MAX}_{\msgid_1} \ldots := D_{\msgid_n} := D^{MAX}_{\msgid_n} := 1$ and a per message-delay $\delay_{\msgid_i} = \delay$ for $i \in [n]$.
                  \item Set  $\vecM := \vecM \concat (m, \msgid_1, D_{\msgid_1}, \party_1) \concat \ldots \concat (m, \msgid_n, D_{\msgid_n}, \party_n)$.
                  \item Send $(\textsc{diffuse}, \sid, m, \party_s, (\party_1, \msgid_1), \ldots ,(\party_n, \msgid_n))$ to the adversary.
            \end{cccEnum}

            \item Upon receiving $(\textsc{fetch}, \sid)$ from $\party \in \partyset$, or from \adv on behalf of a corrupted party \party:
            %
            \begin{cccEnum}[nosep]
                  \item For all tuples $(m, \msgid, D_\msgid, \party) \in \vecM$, set $D_\msgid := D_\msgid - 1$.
                  \item Let $\vecM^\party_0$ denote the subvector \vecM including all tuples of the form $(m, \msgid, D_\msgid, \party)$ with $D_\msgid \le 0$ (in the same order as they appear in \vecM).
                  %
                  Delete all entries in $\vecM^\party_0$ from \vecM and in case some $(m, \msgid, D_\msgid, \party)$ is in
                  $\vecM^\party_0$, where \party is honest, set $\delay_{\msgid'} = \delay$ for any $(m, \msgid', D_{\msgid'}, (\cdot, \sid))$ in \vecM and replace this record by $(m, \msgid', \min \{ D_{\msgid'}, \delay \}, \party')$.
                  \item Output $\vecM^\party_0$ to \party (if \party is corrupted, send $\vecM^\party_0$ to \adv).
            \end{cccEnum}
      \end{cccItemize}

      \paragraph{Additional adversarial capabilities:}
      %
      \begin{cccItemize}[nosep]
            \item Upon receiving $(\textsc{diffuse}, \sid, m)$ from some corrupted $\party_s \in \partyset$ (or from \adv on behalf of $\party_s$ if corrupted), 
            % do 
            execute it the same way as an honest-sender diffuse, with the only difference that $\delay_{\msgid_i} = \infty$.

            \item Upon receiving $(\textsc{delays}, \sid, (T_{\msgid_{i_1}}, \msgid_{i_1}), \ldots, (T_{\msgid_{i_\ell}}, \msgid_{i_\ell}))$ from the adversary do the following for each pair $(T_{\msgid_{i_j}}, \msgid_{i_j})$:
            %
            if $D^{MAX}_{\msgid_{i_j}} + T_{\msgid_{i_j}} \le \delay_{\msgid_{i_j}}$ and $\msgid_{i_j}$ is a message-ID of receiver $\party = (\cdot, \sid)$ registered in the current \vecM, set $D_{\msgid_{i_j}} := D_{\msgid_{i_j}} + T_{\msgid_{i_j}}$ and set $D^{MAX}_{\msgid_{i_j}} := D^{MAX}_{\msgid_{i_j}} + T_{\msgid_{i_j}}$; otherwise, ignore this pair.

            \item Upon receiving $(\textsc{swap}, \sid, \msgid, \msgid')$ from the adversary, if \msgid and $\msgid'$ are message-IDs registered in the current \vecM, then swap the triples $(m, \msgid, D_\msgid, (\cdot, \sid))$ and $(m, \msgid', D_{\msgid'}, (\cdot, \sid))$ in \vecM.
            %
            Return $(\textsc{swap}, \sid)$ to the adversary.

            \item Upon receiving $(\textsc{get-reg}, \sid)$ from \adv, return the response $(\textsc{get-reg}, \sid, \partyset)$ to \adv.
      \end{cccItemize}
\end{cccFunctionality}

\paragraph{Random oracle for PoW and its wrapper.}
%
Our random oracle for generating PoW follows that in~\cite{C:BMTZ17} except that now it is a shared functionality.

\begin{cccFunctionality}
    {\funcGRO}
    {global-random-oracle}
    {The global random oracle.}

    The functionality is parameterized by a security parameter $\kappa$.

    \addtocounter{table}{-1}
    \begin{tabularx}{.9\textwidth}{c  X}
        \toprule[.3mm]
        \textbf{State Variable}
         & \textbf{Description}
        \\ \midrule[.3mm]
        $\partyset \gets \emptyset$
         & The set of registered parties.
        \\ \midrule
        $H \gets \emptyset$
         & A dynamically updatable function table where $H[x] = \bot$ denotes the fact that no pair of the form $(x, \cdot)$ is in $H$.
        \\ \bottomrule[.3mm]
    \end{tabularx}

    \begin{cccItemize}[noitemsep]
        \item \textbf{Eval.} Upon receiving $(\textsc{eval}, \sid_{RO}, x)$ from some party $\party \in \partyset$ (or from \adv on behalf of a corrupted \party), do the following:
        %
        \begin{cccEnum}[nosep]
            \item If $H[x] = \bot$ sample a value $y$ uniformly at random from $\{0, 1\}^\kappa$ and set $H[x] \gets y$.
            \item Return $(\textsc{eval}, \sid, x, H[x])$ to the requestor.
        \end{cccEnum}
    \end{cccItemize}
\end{cccFunctionality}

In order to limit the adversary on making a certain number of queries per round, we adopt a functionality wrapper \cite{C:BMTZ17} that wraps the corresponding resource to capture such restrictions.
%
Note that since now \funcGRO is shared among different sessions, the adversary can ask the environment to make queries on behalf.
%
Hence, our wrapper \wrapper{\funcGRO} also puts restrictions on queries that are made by the environment.

\begin{cccFunctionality}
    {\wrapper{\funcGRO}}
    {random-oracle-wrapper}
    {The random oracle wrapper.}

    This functionality maintains state variables as follows.

    \addtocounter{table}{-1}
    \begin{tabularx}{.9\textwidth}{c  X}
        \toprule[.3mm]
        \textbf{State Variable}
         & \textbf{Description}
        \\ \midrule[.3mm]
        $\partyset \gets \emptyset$
         & The set of registered parties; the current set of corrupted parties is denoted by $\partyset'$.
        \\ \midrule
        $\tau \gets 0$
         & The (real-time) clock tick counter.
        \\ \midrule
        $h_\tau$
         & An upper bound which restricts the \func-evaluations of all alert parties at time $\tau$.
        \\ \midrule
        $q_\honestPartySet, q_\adv \gets 0$
         & The alert/adversary evaluation counter.
        \\ \bottomrule[.3mm]
    \end{tabularx}

    \paragraph{Relaying inputs to the random oracle:}
    %
    \begin{cccItemize}[nosep]
        \item Upon receiving $(\textsc{eval}, \sid, x)$ from \adv on behalf of a corrupted party $P \in \partyset'$ or a de-synchronized party \party, first execute \textit{Round Reset}, then do the following.
        %
        \begin{cccEnum}[nosep]
            \item Set $q_\adv \gets q_\adv + 1$.
            \item \textbf{If} $q_\adv \le h_\tau$ \textbf{then} forward the request to \funcGRO and return to \adv whatever \funcGRO returns.
        \end{cccEnum}

        \item Upon receiving $(\textsc{eval}, \sid, x)$ from an alert party \party, first execute \textit{Round Reset}, then do the following.
        %
        \begin{cccEnum}[nosep]
            \item Set $q_\honestPartySet \gets q_\honestPartySet + 1$.
            \item \textbf{If} $q_\honestPartySet \le h_\tau$ \textbf{then} forward the request to \funcGRO and return to \party whatever \funcGRO returns.
            \item \textbf{If} $q_\honestPartySet \ge h_\tau$ \textbf{then} send $(\textsc{clock-update}, \sid_C)$ to \funcClock.
        \end{cccEnum}
    \end{cccItemize}

    \paragraph{Corruption handling:}
    %
    \begin{cccItemize}[nosep]
        \item Upon receiving $(\textsc{corrupt}, \sid, \party)$ from the adversary, set $\partyset' \gets \partyset' \cup \party$.
    \end{cccItemize}

    \medskip\emph{Procedure Round-Reset:}
    %
    Send $(\textsc{clock-read}, \sid_C)$ to \funcClock and receive $(\textsc{clock-read}, \allowbreak \sid_C, \tau')$ from \funcClock. If $|\tau - \tau' | > 0$, then set $q_\honestPartySet, q_\adv \gets 0$ and $\tau \gets \tau'$.
\end{cccFunctionality}


\paragraph{Global random oracle.}
%
The restricted programmable and observable global random oracle \funcGrpoRO follows the modelling in~\cite{EC:CDGLN18}.
%
We allow the adversary to observe and program the GRO; meantime, every party (in the same session) can check if a point is programmed by calling the ``IsProgrammed'' interface.

\begin{cccFunctionality}
      {\funcGrpoRO}
      {global-random-oracle-rpo}
      {The restricted global random oracle with programmability and observability.}

      The functionality is parameterized by the security parameter $\kappa$.
      %
      It maintains a dynamically updatable list $\mathsf{H}$ and $\mathsf{prog}$
      %
      Initially, $\mathsf{H} = \mathsf{prog} = \emptyset$.

      \begin{enumerate}[label=\FlatSteel, leftmargin=*, nosep]
            \item \textbf{Eval.} Upon receiving $(\textsc{eval}, m)$ from a party $(\party, \sid)$ or from the adversary, do the following:
                  %
                  \begin{enumerate}[label=\arabic*., leftmargin=*, nosep]
                        \item If $\mathsf{H}[m] = \bot$, sample a value $h$ uniformly at random from $\{0, 1\}^\kappa$ and set $\mathsf{H}[m] \gets h$.
                        \item Parse $m$ as $(s, m')$.
                        \item If the query is made by the adversary, or if $s \neq \sid$, then add $(s, m', h)$ to the (initailly empty) list of illegitimate queries $\mathcal{Q}_s$.
                        \item Output $(\textsc{eval}, m, h)$ to the requestor.
                  \end{enumerate}

            \item \textbf{Observe.} Upon receiving $(\textsc{observe}, \sid)$ from the adversary: If $\mathcal{Q}_\sid$ does not exist, set $\mathcal{Q}_\sid = \emptyset$. Output $(\textsc{observe}, \mathcal{Q}_\sid)$ to the adversary.

            \item \textbf{Program.} Upon receiving $(\textsc{program}, m, h)$ with $h \in \{0, 1\}^\kappa$ from the adversary, do the following:
                  %
                  \begin{enumerate}[label=\arabic*., leftmargin=*, nosep]
                        \item If $\exists h' \in \{0, 1\}^\kappa$ such that $\mathsf{H}[m] = h'$ and $h \neq h'$, ignore this input.
                        \item Set $\mathsf{H}[m] \gets h$ and $\mathsf{prog} \gets \mathsf{prog} \cup \{ m \}$.
                        \item Output $(\textsc{program}, \ok)$ to the adversary.
                  \end{enumerate}

            \item \textbf{IsProgrammed.} Upon receiving $(\textsc{is-programmed}, m)$ from a party $(\party, \sid)$ or from the adversary, do the following:
                  %
                  \begin{enumerate}[label=\arabic*., leftmargin=*, nosep]
                        \item If the input was given by $(\party, \sid)$, parse $m$ as $(s, m')$. If $s \neq \sid$, ignore this input.
                        \item Set $b \gets m \in \mathsf{prog}$ and output $(\textsc{is-programmed}, b)$ to the requestor.
                  \end{enumerate}
      \end{enumerate}
\end{cccFunctionality}

\paragraph{Non-interactive zero knowledge scheme \NIZK.}
%
The NIZKPoK scheme that we use in this paper are from \cite{TCC:LysRos22}, specifically a non-interactive, straight-line extractable (NISLE) proof
system satisfying the following three properties.

\begin{definition}
    [Overwhelming Completeness]
    \label{def:overwhelming-completeness}

    A NISLE proof system $\Pi^{\mathtt{SLC}}_R = (\mathtt{Setup}^H, \mathtt{Prove}^H,\allowbreak \mathtt{Verify}^H, \mathtt{SimSetup}, \mathtt{SimProve}, \mathtt{Extract})$ for relation $R$ in the random-oracle model has the \emph{overwhelming completeness} property if for any security parameter $\lambda$, any random oracle $H$, any $(x, w) \in R$, and any proof $\pi \gets \Pi^{\mathtt{SLC}}_R.\mathtt{Prove}^H(x, w)$,
    %
    \( \Pr[\Pi^{\mathtt{SLC}}_R.\mathtt{Verify}^H(x, \pi) = 1] \ge 1 - \negl(\lambda). \)
\end{definition}

\begin{definition}
    [Non-Interactive Multiple SHVZK]
    \label{def:non-interactive-multiple-SHVZK}
    
    A NISLE proof system $\Pi^{\mathtt{SLC}}_R = (\mathtt{Setup}^H,\allowbreak \mathtt{Prove}^H,\allowbreak \mathtt{Verify}^H,\allowbreak \mathtt{SimSetup},\allowbreak \mathtt{SimProve},\allowbreak \mathtt{Extract})$ for relation $R$ in the random-oracle model has the \emph{non-interactive multiple special honest-verifier zero-knowledge (NIM-SHVZK)} property if for any security parameter $\lambda$, any random oracle $H$, any PPT adversary \adv, and a bit $b \overset{\$}{\gets} \{0, 1\}$, ther exist some negligible function \negl such that $\Pr[b' = b] \le 1 /2 + \negl(\lambda)$, where $b'$ is the result of running the game $\mathsf{NIM}\text{-}\mathsf{SHVZK}^{H_*, *}_{\adv, \Pi^{\mathtt{SLC}}_R} (1^\lambda, b)$.
    %
    We say \adv wins the $\mathsf{NIM}\text{-}\mathsf{SHVZK}$ game if $\Pr[b' = b] > 1 / 2 + \negl(\lambda)$.
\end{definition}

\begin{definition}
    [Non-Interactive Special Simulation-Soundness]
    \label{def:non-interactive-special-simulation-soundness}
    
    A NISLE proof system $\Pi^{\mathtt{SLC}}_R = (\mathtt{Setup}^H,\allowbreak \mathtt{Prove}^H,\allowbreak \mathtt{Verify}^H,\allowbreak \mathtt{SimSetup},\allowbreak \mathtt{SimProve},\allowbreak \mathtt{Extract})$ for relation $R$ in the random oracle model has the \emph{non-interactive special simulation-soundness} property if for any security parameter $\lambda$, any random oracle $H$, any PPT adversary \adv, there exist some negligible function \negl such that  $\Pr[\mathtt{Fail} \gets \mathsf{NIM}\text{-}\mathsf{SSS}^{H_*, \mathtt{Prog}}_{\adv, \Pi^{\mathtt{SLC}}_R} (1^\lambda)] \le \negl(\lambda)$.
\end{definition}

For completeness, we also describe the security game $\mathsf{NIM}\text{-}\mathsf{SSS}^{H_*, \mathtt{Prog}}_{\adv, \Pi^{\mathtt{SLC}}_R} (1^\lambda)$ related to the non-interactive special simulation-soundness property of the NIZK protocol where $\mathcal{Q}_\adv$ are \adv's queries to the RO (details see~\cite{TCC:LysRos22}).

\begin{cccGame}
    {$\mathsf{NIM}\text{-}\mathsf{SSS}^{H_*, \mathtt{Prog}}_{\adv, \Pi^{\mathtt{SLC}}_R} (1^\lambda)$}
    {nim-sss}
    {The non-interactive special simulation-soundness game.}

    \begin{algorithmic}[1]
        \State{$L \gets \bot$}

        \State{$\mathtt{ppm}, z \gets \Pi^{\mathtt{SLC}}_R.\mathsf{SimSetup}^{\mathtt{prog}_L} (1^\lambda)$}

        \State{$\st \gets \adv^{H_L}(1^\lambda, \mathtt{ppm})$}

        \State{$\mathtt{pflist}, \mathtt{Response} \gets \bot$}

        \While{$\st \neq \bot$}

        \State{$(\mathtt{Query}, \mathcal{Q}_\adv, \st) \gets \adv^{H_L}(\st)$}

        \If{$\mathtt{Query} = (\mathtt{Prove}, x, w)$}
        \If{$R(x, w) = 1$}
        \State{$\Pi \gets \Pi^{\mathtt{SLC}}_R.\mathsf{SimProve}^{\mathtt{prog}_L} (z, x)$}

        \State{Append $(x, \pi)$ to $\mathtt{pflist}$}
        \State{Set $\mathtt{Response} \gets (x, \pi)$}
        \EndIf
        \ElsIf{$\mathtt{Query} = (\mathtt{Challenge}, x, \pi)$}
        \If{$\Pi^{\mathtt{SLC}}_R.\mathsf{Verify}^{\mathtt{prog}_L} (x, \pi) = 1$ \textbf{and} $(x, \pi) \not\in \mathtt{pflist}$}
        \State{$w \gets \Pi^{\mathtt{SLC}}_R.\mathsf{Extract}(x, \pi, \mathcal{Q}_\adv)$}
        
        \If{$R(x,w) = 0$}
        \State{\Return $\mathtt{Fail}$}
        \EndIf

        \EndIf
        \EndIf

        \EndWhile

        \State{\Return $\mathtt{Success}$}
    \end{algorithmic}
\end{cccGame}

