\subsection{Sender Order Fairness}
\label{subsec:sender-order-fairness}

Sender-side order fairness captures the natural desire that a ledger \funcFairLedger serializes transactions based on the time that they are released by their senders.
%
This order is well-defined by looking at the \funcFairLedger transaction buffer and order transactions based on their arrival time recorded.
%
Hence, every time the adversary tries to propose a transaction \tx that is sent at time $\tau$ (i.e., \funcFairLedger receives \tx and records $(\tx, \txid, \tau, \party)$), our \textsf{ValidOrder} algorithm in \textsf{SenderOrder}  checks whether there exists valid transactions in \buffer with time $\tau' < \tau$.
%
If such transaction exists, the proposed order is invalid and this will force the $\mathsf{ExtendPolicy}$ to return a default extension.

Regarding the \textsf{DefaultExtension} mechanism, since the original construction in~\cite{C:BMTZ17} has been proposing a default order that follows the one stored in the ledger, we follow that construction with one minor modification.
%
Precisely, in~\cite{C:BMTZ17} a default extension will blockify all pending transactions in the buffer, while ours only include those that are about to violate liveness --- again this adaption follows our blockchain scheme that decouples the consensus and transaction inclusion.
%
We note that this adaption is insignificant, as a good simulator shall never allow the ledger to trigger this subroutine.

\begin{cccAlgorithm}
    {\textsf{SenderOrder}}
    {exact-order}
    {The \textsf{ValidOrder} predicate and \textsf{DefaultExtension} algorithm for sender fairness.}

    \begin{algorithmWithNumbering}{senderOrderCounter}
        \Statex{\underline{$\mathsf{ValidOrder}(\tx, \state, \buffer)$}}

        \State{Parse \tx as $(\tx, \txid, \tau, \party)$}

        \LineComment{Extract the most recent transaction time in the ledger state.}
        \State{$\tau^* \gets \max \{\tau' \mathbin| \BTX = (\tx, \cdot, \tau', \cdot) \wedge \tx \in \state\}$}

        \State{$\vec{\tx} \gets \{\tx \mathbin | \BTX = (\tx, \cdot, \tau', \cdot) \in \buffer \wedge \BTX \not\in \txSeqProposal \wedge \tau^* \le \tau' < \tau \}$}

        \For{$\tx \in \vec{\tx}$}
        \oneLineIf{$\mathsf{ValidTX}(\tx, \state) = \true$}{\Return \false}
        \EndFor
        \State{\Return \true}
    \end{algorithmWithNumbering}

    \medskip

    \begin{algorithmWithNumbering}{senderOrderCounter}
        \Statex{\underline{$\mathsf{DefaultExtension}(\honestInputSeq, \state, \mathtt{NxtBC}, \buffer, \timedState)$}}

        \State{$\ledgerTime \gets$ current ledger time (computed from \honestInputSeq)}

        \State{$\txSeqProposalDefault \gets \tx^{\mathrm{base-tx}}_{\mathrm{minerID}}$ of an honest miner}

        \State{\revisedText{Let $\vec{\tx} = \{\tx \mathbin| \BTX = (\tx, \txid, \ledgerTime', \party) \in \buffer \wedge \ledgerTime' < \ledgerTime - \mathtt{waitTime} \}$}}

        \State{Sort $\vec{\tx}$ according to timestamps}

        \State{$\st \gets \blockify(\txSeqProposalDefault)$}

        \Repeat

        \State{Let $\vec{\tx} = (\tx_1, \ldots, \tx_\ell)$ be the current set of (remaining) transactions}
        \For{$i = 1$ \textbf{to} $\ell$}
        \If{\revisedText{$\mathsf{ValidTX}(\tx_i, \state \concat \st) = \true$ \textbf{and} $\mathsf{ValidOrder}(\tx_i, \state \concat \st ,\buffer) = \true$}}
        \State{$\txSeqProposalDefault \gets \txSeqProposalDefault \concat \tx_i$}
        \State{Remove $\tx_i$ from $\vec{\tx}$}
        \State{$\st \gets \blockify(\txSeqProposal)$}
        \EndIf
        \EndFor

        \Until{\txSeqProposalDefault does not increase any more}

        \oneLineIf{$|\state| + 1 \ge \mathtt{windowSize}$}{$\tauLow \gets \timedState[|\state| - \mathtt{windowSize} + 2]$ \textbf{else} set $\tauLow \gets 0$}

        \State{$c \gets 1$}
        \While{$\ledgerTime - \tauLow > \mathtt{maxTime_{window}}$}
        \State{Set $\txSeqProposal_c \gets \tx^{\mathrm{base-tx}}_{\mathrm{minerID}}$ of an honest miner}
        \State{$\txSeqProposalDefault \gets \txSeqProposalDefault \concat \txSeqProposal_c$}
        \State{$c \gets c + 1$}

        \oneLineIf{$|\state| + c \ge \mathtt{windowSize}$}{$\tauLow \gets \timedState[|\state| - \mathtt{windowSize} + c + 1]$ \textbf{else} $\tauLow \gets 1$}
        \EndWhile

        \State{\Return \txSeqProposalDefault}
    \end{algorithmWithNumbering}
\end{cccAlgorithm}

The sender fairness \textsf{SenderOrder} is our desideratum; unfortunately, \textsf{SenderOrder} is impossible to realize, even in the synchronous network (i.e., $\delay = 1$ in $\funcDiffuse^\delay$).
%
The high level intuition is that transactions issued by corrupted parties can always reach honest parties in two consecutive rounds by sending to one honest party \party at round $r$ and let \party diffuse it to others at round $r + 1$, where in the case of honest party issuing transactions, the sender can book-keep their receiving time to the next round.
%
The fact that the adversary \adv can exploit this one round advantage allows him to make the following two executions identical: in the first execution, \adv sends a transaction \tx at round $r$ to only one honest party and in the second execution he sends \tx to all honest parties at round $r + 1$.
%
Later, in the second execution \adv lets one corrupted party behave like it receive and diffuses the message at round $r$.
%
In the first execution, \tx in \funcFairLedger owns timestamp $r$ while in the second execution it owns timestamp $r + 1$, however these two executions are indistinguishable.
