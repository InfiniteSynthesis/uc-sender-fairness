\subsection{Approximate Sender Order Fairness}
\label{subsec:approximate-order-fairness}

Given that \textsf{SenderOrder} is unachievable unless under unrealistic network assumptions (that is, the delivery of a transaction is instant and before any transaction that is issued later thus every honest parties see the same sequence of transactions), we next propose a natural alternative that relaxes the sender order fairness, giving up on asking for the order of two transactions if they enter the network at approximately the same time.

We leave the \textsf{DefaultExtension} procedure the same as that in \textsf{SenderOrder} where an order following what \funcFairLedger receives is proposed (hence we omit it in the algorithm description).
%
Note that such default extension will still yield a legitimate order with  respect to $\delta$-\textsf{ApproxSenderOrder}.
%
Regarding the \textsf{ValidOrder} algorithm, when the adversary is to propose a transaction \tx sending the ledger at time $\tau$, all transactions in the buffer which are sent at time before $\tau - \delta$ and are valid with respect to state should be proposed before \tx; otherwise, \textsf{ValidOrder} returns \false to instruct the ledger to use the \textsf{DefaultExtension}.

\begin{cccAlgorithm}
    {$\delta$-\textsf{ApproxSenderOrder}}
    {delta-approximate-order}
    {The \textsf{ValidOrder} predicate for approximate sender fairness.}

    \begin{algorithmic}[1]
        \Statex{\underline{$\mathsf{ValidOrder}(\tx, \txSeqProposal, \buffer)$}}
        \State{Parse \tx as $(\tx, \txid, \tau, \party)$}

        \LineComment{Extract the most recent transaction time in the ledger state.}
        \State{$\tau^* \gets \max \{\tau' \mathbin| \BTX = (\tx, \cdot, \tau', \cdot) \wedge \tx \in \state\}$}

        \State{$\vec{\tx} \gets \{\BTX = (\cdot, \cdot, \tau', \cdot) \mathbin| \BTX \in \buffer \wedge \BTX \not\in \txSeqProposal \wedge \tau^* \le \tau' < \tau - \delta\}$}

        \For{$\tx \in \vec{\tx}$}
        \oneLineIf{$\mathsf{ValidTX}(\tx, \state) = \true$}{\Return \false}
        \EndFor
        \State{\Return \true}
    \end{algorithmic}
\end{cccAlgorithm}

\paragraph{Impossibility of $\delta$-\textsf{ApproxSenderOrder} with \delay-bounded network delay for $\delta < \delay$.}
%
We show that in the same setting as Bitcoin (i.e., with \funcClock, \wrapper{\funcGRO} and \funcDiffuse), it is impossible to guarantee fairness for two transactions $\tx, \tx'$ that enters the system less than $\delta$ rounds apart from each other for any $\delta < \delay$.

\begin{theorem} \label{thm:impossibility-bitcoin-setting}
	In the $(\funcClock, \wrapper{\funcGRO}, \funcDiffuse^\delay)$-hybrid environment, there exist no protocol $\Pi$ that securely realizes \funcFairLedger parameterized with $\delta$-\textsf{ApproxSenderOrder} for any $\delta < \delay$.
\end{theorem}

\begin{proof}
	Suppose towards a contradiction, there exists a protocol $\Pi$ that securely realizes \funcFairLedger parameterized with $\delta$-\textsf{ApproxSenderOrder} in the $(\funcClock, \wrapper{\funcGRO}, \funcDiffuse^\Delta)$-hybrid environment.

	Consider two transactions $\tx, \tx'$.
	%
	Let $\tau, \tau'$ denote the timestamp that \funcFairLedger assigns to them in the transaction buffer respectively.
	%
	Consider a world $W_1$ where \tx is sent more than $\delta$ yet less than \delay rounds before $\tx'$ (i.e., $\tau + \delta < \tau' < \tau + \delay$).
	%
	I.e., $\delta$-\textsf{ApproxSenderOrder} would force \funcFairLedger to output $\tx \prec \tx'$.
	%
	Regarding the transaction diffusion pattern in $W_1$, assume all but the sender of \tx receives \tx at time $\tilde{\tau}$ such that $\tau' < \tilde{\tau} < \tau + \delay$;
	%
	and all but the sender of $\tx'$ receives $\tx'$ at time $\tilde{\tau}'$ such that $\tau' \le \tilde{\tau}' < \tilde{\tau}$.
	%
	I.e., for all parties except the senders, they saw $\tx' \prec \tx$
	unanimously.
	%
	Then, we consider the following three scenarios.

	First, neither the sender of \tx nor the sender of $\tx'$ has the chance to send a message.
	%
	We prove that for all other protocol participants in $W_1$ this is indistinguishable from the world $W_2$ where \funcFairLedger shall output $\tx' \prec \tx$.
	%
	Consider the following scenario in $W_2$: two transactions $\tx, \tx'$ are sent at time $\tau', \tau$ respectively and are delivered to all other parties at time  $\tilde{\tau}$, $\tilde{\tau}'$ respectively.
	%
	Due to $\delta$-\textsf{ApproxSenderOrder}, \funcFairLedger outputs $\tx' \prec \tx$.
	%
	The local views of all other parties are identical to the ones in $W_1$ hence the messages they send are also identical.

	Second, at least one of the two senders of $\tx, \tx'$ successfully sends at least one message.
	%
	Note that this implies $\Pi$ is a protocol such that the pairwise order of any two transactions could be dominated by a single party (or, a tiny fraction of resources to run the protocol).
	%
	Again we show that $W_1$ is indistinguishable from the world $W_3$ where \funcFairLedger shall output $\tx' \prec \tx$.
	%
	Suppose, with out loss of generality, \tx is generated by a corrupted party.
	%
	Consider the following scenario in $W_3$: the transaction timestamp in \funcFairLedger and diffusion pattern are exactly the same as those in $W_2$ except that now the corrupted issuer of \tx sends a message claiming that \tx is sent at time $\tau$ (this is a valid message even if \delay is known and parties use it to filter old messages).
	%
	If in $W_1$ the issuer of \tx sends the same message that \tx is generated at time $\tau$, then in all honest parties' local views, $W_3$ is identical to the scenario in $W_1$ and they send the identical messages.

	Finally, consider the case where both transaction issuers send messages claiming their transaction time.
	%
	We show $W_1$ is indistinguishable from the world $W_4$ where \funcFairLedger shall output $\tx' \prec \tx$.
	%
	Consider the following scenario in $W_4$: the transaction issuer of \tx is corrupted and \funcFairLedger assigns timestamp $\tau', \tau - \epsilon$ to transaction \tx and $\tx'$ respectively.
	%
	In addition, $\tx, \tx'$ are diffused to all parties except their senders at at the same time as those in $W_1$; the (honest) issuer of $\tx'$ sends messages saying that $\tx'$ is sent at time $\tau - \epsilon$ and the corrupted issuer of \tx sends messages claiming \tx is sent at time $\tau$.
	%
	Assume in $W_1$, the issuer of $\tx'$ is corrupted; then, the honest issuer of \tx sends messages claiming that \tx is sent at time $\tau$ and the corrupted issuer of $\tx'$ sends messages claiming that $\tx'$ is sent at time $\tau - \epsilon$.
	%
	In such a case, the diffusion pattern and messages sent in both $W_1$ and $W_4$ are identical.
\end{proof}

\paragraph{Approximate order with any private funcitonality \F.}
%
Trusted hardware may be considered as a promising solution to transaction order fairness, as they are promised to help ``authenticate and timestamp'' a transaction that is generated even by a malicious host.
%
Despite the fact that trusted execution environments are hard to synchronize, we show that when network delay exists, it leads to the imprecision on order fairness that cannot be circumvented by any powerful time-aware private functionalities.

We consider the model where there exists a functionality $\F_i$ for each party $\party_i$ which, can only communicate with $\party_i$, global clock \funcClock and a correlated randomness functionality $\mathcal{G}_{\mathsf{CR}}$; however $\F_i$ can function arbitrarily.
%
Specifically, the correlated randomness setup $\mathcal{G}_{\mathsf{CR}}$ returns a random string $s$ from some distribution $\mathcal{D}$ to a functionality \F only when \F owns the code as specified in $\mathsf{code}$ in $\mathcal{G}_{\mathsf{CR}}$; in such a way only the designated \F can acquire the randomness.
%
This can be achieved, by considering the request message of format (cf. UC with global subroutine in~\cite{TCC:BCHTZ20}) $(\mu, (\sid, \pid))$ where $\mu$ specifies the sender's code.

\begin{cccFunctionality}
    {$\mathcal{G}_{\mathsf{CR}}$}
    {correlated-randomness}
	{Correlated randomness.}

    This functionality is parameterized by a distribution $\mathcal{D}$.

	\begin{cccItemize}[noitemsep]
		\item Upon receiving $\mathsf{eid} = (\mu, (\sid, \pid))$ from \party:
		%
		\begin{cccEnum}[nosep]
			\item If $\mu \neq \mathsf{code}$ ignore this message.

			\item If there is no tuple of the form $(\sid, \ldots)$, generate $(s_1, \ldots, s_n) \gets \mathcal{D}(1^\lambda)$ and store $(\sid, s_1, \ldots, s_n)$. Otherwise, retrieve $(\sid, s_1, \ldots, s_n)$.

			\item Send Send $(\sid, s_i)$ to $\party_i$.
		\end{cccEnum}
	\end{cccItemize}
\end{cccFunctionality}

Given $\mathcal{G}_{\mathsf{CR}}$, all responses parties acquired from \F can be verified by other parties (in other words, they can be ``convinced'' by the messages from \F).
%
We showcase that even if \F can implement an arbitrary function and is time-aware (i.e., \F learns the global time from \funcClock), this does not help circumvent the impossibility result from the network delay.
%
The subtlety in our proof, at a high level, is that corrupted parties can keep querying his local \F for the same message in every round via different corrupted parties.
%
We formalize this result in~\cref{theorem:impossibility-with-f}.

\begin{theorem}
	[Impossibility of $\delta$-approximate order with any private \F]
	\label{theorem:impossibility-with-f}
	
	If $\delay > 0$, then for any $\delta < \delay$ there exist no protocol $\Pi$ that securely realizes \funcFairLedger parameterized with $\delta$-\textsf{ApproxSenderOrder} in the $(\funcClock, \wrapper{\funcGRO}, \funcDiffuse,\allowbreak \mathcal{G}_{\mathsf{CR}}, \mathcal{F})$-hybrid environment.
\end{theorem}

\begin{proof}
	Suppose towards a contradiction, there exists a protocol $\Pi$ that securely realizes \funcFairLedger parameterized with $\delta$-\textsf{ApproxSenderOrder} in the $(\funcClock, \wrapper{\funcGRO},\allowbreak \funcDiffuse, \mathcal{G}_{\mathsf{CR}}, \F)$-hybrid environment.

	Consider two transactions $\tx, \tx'$ in scenario $W_1$.
	%
	Let $\tau, \tau'$ denote the timestamp that \funcFairLedger assigns to them in the transaction buffer respectively.
	%
	With out loss of generality, assume $\tau + \delta < \tau' < \tau + \delay$ and $\tx'$ is generated by the adversary.
	%
	Due to $\delta$-\textsf{ApproxSenderOrder}, \funcFairLedger should output $\tx \prec \tx'$.
	%
	We show this is indistinguishable from the scenario $W_2$ where \funcFairLedger shall output $\tx' \prec \tx$.
	%
	Before we go to $W_2$, consider the responses from \F in $W_1$.
	%
	The issuer of \tx, when sending \tx, can attach the message that he queries his own \F at round $\tau$; the (corrupted) issuer of $\tx'$, when sending \tx', however can attach the message that he queries his own \F at round $\tau$ (but not $\tau'$).
	%
	This is because the adversary can keep querying $\tx'$ to \F in every round up to $\tau'$ (which includes round $\tau < \tau'$).
	%
	Since in each round, the adversary can query \F via different corrupted parties (and these parties behave honestly for all other operations) the message that he acquired from \F in round $\tau $ is admissible by all honest parties.

	Now consider the following scenario $W_2$: \funcFairLedger assigns timestamp $\tau', \tau$ to transaction \tx and $\tx'$ respectively, and the transaction issuer of \tx is corrupted.
	%
	In addition, $\tx, \tx'$ are diffused to all parties except their senders at at the same time in $W_1$ and $W_2$ (this diffusion pattern is possible, see the proof of~\cref{thm:impossibility-bitcoin-setting}).
	%
	In $W_2$, the (honest) issuer of $\tx'$ queries its private \F the same time as it was sent messages hence get a response with respect to time $\tau$; meanwhile the corrupted issuer of \tx keeps querying \tx and diffuses \tx with a response with respect to time $\tau$ at round $\tau'$.

	We learn that the diffusion pattern and the responses from \F sent in both scenarios are identical, thus parties must decide the same order, which contradicts the fact that the protocol realizes $\delta$-\textsf{ApproxSenderOrder}.
\end{proof}