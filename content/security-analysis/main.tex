\section{Security Analysis}
\label{sec:security-analysis}

\paragraph{Overview of the security analysis.}
%
In the ideal world, the simulator \simulator simulates the network functionality for a black-box adversary \adv; \simulator also simulates the ledger maintenance procedure of honest parties and jointly builds a blockchain with \adv.
%
When dummy parties issue new transactions, \simulator generates a fake ciphertext of an all-zero string and send it to the simulated network.
%
When an encrypted transaction is received from \adv, \simulator extracts the plain transaction by using \NIZK.
%
Importantly, when the simulated blockchain has progressed such that \simulator learns the transaction order, \simulator proposes those transactions (using \txid) to \funcFairLedger and thus learn the honest transactions.
%
\simulator then programs the \funcGrpoRO (and hides this ``isProgramed'' information to \adv) and equivocate the corresponding encrypted transactions to the real honest transaction.

We provide some more details on handling honest transactions.
%
When the ledger \funcFairLedger notifies \simulator that a dummy honest party \party submits a new transaction of length $m$ with \txid at time $\tau$, \simulator simulates the $\mathsf{IssueNewTransaction}$ procedure for \party by associating a block \block with this transaction, based on \party's simulated local chain state (see \textsc{IssueNewTransaction} in $\mathcal{S}_{\mathsf{ledger}}$ Part~\ref{simulator:internal-procedures}).
%
In order to learn the honest transactions, \simulator waits until that in the simulated blockchain, all transactions that might precede the transaction with \txid get settled.
%
\simulator proposes the sequence of transactions to \funcFairLedger and thus learns the transaction \tx that is associated with \txid.
%
Note that since the transaction decryption interval parameter $\Lambda$ is set slightly longer than the time that it will get settled on the blockchain, \simulator always has enough time to program \funcGrpoRO and equivocate the ciphertext of an all-zero string to \tx.
%
We detail this part in \textsc{ExtendLedgerState} in $\mathcal{S}_{\mathsf{ledger}}$ Part~\ref{simulator:internal-procedures}.

The simulator terminates when certain bad event happens.
%
Specifically, we define the following bad events \badCP (violation of common prefix), \badCQ (violation of chain quality), \badCG (violation of chain growth), \badProfile (violation of honest profile majority in any \PBWindowLen consecutive blocks), \badNIZK (failure of the NIZK scheme), \badDec (failure of transaction equivocation), \badTicket (failure of selecting enclave public-key subset with at least one honest key).
%
We prove that all these bad events happen with negligible probability with respect to the security parameter.
%
Hence, the simulator can run the simulation for any polynomial time without abortion.

\paragraph{The simulator.}
%
We present the simulator used in the UC proof of \protocFairLedger that securely implements the ledger functionality \funcFairLedger.

\begin{simulatorbox}
    {$\mathcal{S}_{\mathsf{ledger}}$ (Part 1 - Main structure)}
    {main-structure}

    \paragraph{Overview:}
    %
    \begin{cccItemize}[nosep]
        \item The simulator internally emulates all local UC functionalities by running the code (and keeping the state) of \funcGRO, $\funcDiffuse^{\mathsf{bc}}$, $\funcDiffuse^{\mathsf{tx}}$ and $\funcDiffuse^{\mathsf{pb}}$.

        \item The simulator mimics the execution of \protocFairLedger for each honest party $U_p$ (including their state and the interaction with the hybrids).

        \item The simulator emulates a view towards the adversary \adv in a black-box way, i.e., by internally running adversary \adv and simulating his interaction with the protocol (and hybrids) as detailed below for each hybrid.
        %
        To simplify the description, we assume \adv does not violate the requirements by the wrapper \wrapper{\funcGRO}.

        \item For global functionalities \funcGRO and \funcEnclave, the simulator simply relays the messages sent from \adv to the
        global functionalities (and returns the generated replies).
        %
        Recall that the ideal world consists of the dummy parties, the ledger functionality, the clock, the global random oracles and the enclave.
    \end{cccItemize}

    \paragraph{Party sets:}
    %
    An honest miner \party registered to \funcFairLedger is assumed to be registered in all simulated functionalities.
    %
    Upon any activation, the simulator will query the current party set from the ledger (and simulate the corresponding message they send out to the network in the first maintain-ledger activation after registration), query all activations from honest parties \honestInputSeq, and read the current clock value to learn the time.
    %
    In particular, the simulator knows which parties are honest and synchronized and which parties are de-synchronized.

    \paragraph{Messages from the clock:}
    %
    Upon receiving $(\textsc{clock-update}, \sid_C , U_p)$ from \funcClock, if $U_p$ is an honest registered party, then remember that this party has received such a clock update (and the environment gets an activation).
    %
    Otherwise, send $(\textsc{clock-update}, \sid_C , U_p)$ to \adv.

    \paragraph{Messages from the ledger:}
    %
    \begin{cccItemize}[nosep]
        \item Upon receiving $(\textsc{submit}, m, \txid, \tau, \party)$ from \funcFairLedger where $m$ denotes the transaction length, execute $\textsc{IssueNewTransaction}(\party, \tau, m)$ and denote response by $(\pi, ct, \ticket)$.
        %
        Forward $(\textsc{diffuse}, \sid,\allowbreak (\pi, ct, \ticket))$ to the simulated network $\funcDiffuse^{\mathsf{tx}}$ in the name of \party.
        %
        Output the answer of $\funcDiffuse^{\mathsf{tx}}$ to the adversary.

        \item Upon receiving $(\textsc{maintain-ledger}, \sid, \mathrm{minerID})$ from \funcFairLedger, extract from \honestInputSeq the party \party that issued this query.
        %
        If \party has already completed its round-task, then ignore this request.
        %
        Otherwise, execute $\textsc{SimulateMining}(\party, \ledgerTime)$.
    \end{cccItemize}

    \paragraph{Messages from \funcGrpoRO:}
    %
    \simulator acts as a dummy adversary and forward all responses from \funcGrpoRO to its requestor except that when receiving a $(\textsc{is-programmed}, m)$ request, return $(\textsc{is-programmed}, \false)$ if the $m$ is previously programed by the simulator.
\end{simulatorbox}
\begin{simulatorbox}
    {$\mathcal{S}_{\mathsf{ledger}}$ (Part 2 - Black-Box Interaction)}
    {black-box-interaction}

    \newcommand*{\msgid}{\mathEnv{\mathsf{mid}}}

    \emph{Simulation of the Network $\funcDiffuse^{\mathsf{bc}}$ (over which chains are sent) towards \adv:}
    %
    \begin{cccItemize}[nosep]
        \item Upon receiving $(\textsc{diffuse}, \sid, (\chain_{i_1}, U_{i_1}), \ldots, (\chain_{i_\ell}, U_{i_\ell}))$ with a list of chains and corresponding parties from \adv (or on behalf some corrupted $\party \in \partyset$), then do the following:
        %
        \begin{cccEnum}[nosep]
            \item Relay this input to the simulate network functionality and record its response to \adv.
            \item Execute $\textsc{ExtendLedgerState}(\ledgerTime)$.
            \item Provide \adv with the recorded output of the simulated network.
        \end{cccEnum}

        \item Upon receiving $(\textsc{diffuse}, \sid, \chain)$ from \adv on behalf of some \emph{corrupted} party \party, then do the following:
        %
        \begin{cccEnum}[nosep]
            \item Relay this input to the simulate network functionality and record its response to \adv.
            \item Execute $\textsc{ExtendLedgerState}(\ledgerTime)$.
            \item Provide \adv with the recorded output of the simulated network.
        \end{cccEnum}

        \item Upon receiving $(\textsc{fetch}, \sid)$ from \adv on behalf some corrupted $\party \in \partyset$ forward the request to the simulated $\funcDiffuse^{\mathsf{bc}}$ and return whatever is returned to \adv.

        \item Upon receiving $(\textsc{delays}, \sid, (T_{\msgid_{i_1}}, \msgid_{i_1}), \ldots, (T_{\msgid_{i_\ell}}, \msgid_{i_\ell}))$ from \adv forward the request to the simulated $\funcDiffuse^{\mathsf{bc}}$ and record the answer to \adv.
        %
        Before giving this answer to \adv, query the ledger state \state and execute $\textsc{AdjustView}(\state, \ledgerTime)$.

        \item Upon receiving $(\textsc{swap}, \sid, \msgid, \msgid')$ from \adv forward the request to the simulated $\funcDiffuse^{\mathsf{bc}}$ and record the answer to \adv.
        %
        Before giving this answer to \adv, query the ledger state \state and execute $\textsc{AdjustView}(\state, \ledgerTime)$.
    \end{cccItemize}

    \medskip\emph{Simulation of the Network $\funcDiffuse^{\mathsf{tx}}$ (over which transactions are sent) towards \adv:}
    %
    \begin{cccItemize}[nosep]
        \item Upon receiving $(\textsc{diffuse}, \sid, \tx = (\pi, ct, \tau))$ from \adv (or on behalf some corrupted $\party \in \partyset$), then do the following:
        %
        \begin{cccEnum}[nosep]
            \item If $\tau$ is a valid ticket with respect to an existing block that only contains corrupted enclave public keys, \textbf{Abort} simulation: violation of good ticket (event \badTicket)
            \item Extract \tx from $\pi$ using $\NIZK.\mathsf{SimSetup}$, $\NIZK.\mathsf{SimProve}$ and $\NIZK.\mathsf{Extract}$. If extraction fails, \textbf{Abort} simulation: violation of good NIZK (event \badNIZK)
            \item Submit \tx to the ledger on behalf of this corrupted party, and receive for the transaction id \txid.
            \item Forward the request to the internally simulated $\funcDiffuse^{\mathsf{tx}}$, which replies for each message with a message-ID \msgid.
            \item Remember the association between \msgid and the corresponding \txid.
            \item Provide \adv with whatever the network outputs.
        \end{cccEnum}

        \item Upon receiving $(\textsc{fetch}, \sid)$ from \adv on behalf some corrupted $\party \in \partyset$ forward the request to the simulated $\funcDiffuse^{\mathsf{tx}}$ and return whatever is returned to \adv.

        \item Upon receiving $(\textsc{delays}, \sid, (T_{\msgid_{i_1}}, \msgid_{i_1}), \ldots (T_{\msgid_{i_\ell}}, \msgid_{i_\ell}))$ from \adv forward the request to the simulated $\funcDiffuse^{\mathsf{tx}}$ and return whatever is returned to \adv.

        \item Upon receiving $(\textsc{swap}, \sid, \msgid, \msgid')$ from \adv forward the request to the simulated $\funcDiffuse^{\mathsf{tx}}$ and return whatever is returned to \adv.
    \end{cccItemize}

    \medskip\emph{Simulation of the Network $\funcDiffuse^{\mathsf{pb}}$ ($\funcDiffuse^\pk$ resp.) (over which input blocks [enclave public keys resp.] are sent) towards \adv:}

    \begin{cccItemize}[nosep]
        \item Upon receiving $(\textsc{diffuse}, \sid, m)$ from \adv with an input block $m$ on behalf some corrupted $\party \in \partyset$, then do the following:
        %
        \begin{cccEnum}[nosep]
            \item Forward the request to the internally simulated $\funcDiffuse^{\mathsf{pb}}$ ($\funcDiffuse^\pk$ resp.), which replies for each message with a message-ID \msgid.
            \item Remember the association between \msgid and the corresponding input block.
            \item Provide \adv with whatever the network outputs.
        \end{cccEnum}

        \item Upon receiving $(\textsc{fetch}, \sid)$ from \adv on behalf some corrupted $\party \in \partyset$ forward the request to the simulated $\funcDiffuse^{\mathsf{pb}}$ ($\funcDiffuse^\pk$ resp.) and return whatever is returned to \adv.

        \item Upon receiving $(\textsc{delays}, \sid, (T_{\msgid_{i_1}}, \msgid_{i_1}), \ldots (T_{\msgid_{i_\ell}}, \msgid_{i_\ell}))$ from \adv forward the request to the simulated $\funcDiffuse^{\mathsf{pb}}$ ($\funcDiffuse^\pk$ resp.) and return whatever is returned to \adv.

        \item Upon receiving $(\textsc{swap}, \sid, \msgid, \msgid')$ from \adv forward the request to the simulated $\funcDiffuse^{\mathsf{pb}}$ ($\funcDiffuse^\pk$ resp.) and return whatever is returned to \adv.
    \end{cccItemize}
\end{simulatorbox}
\begin{simulatorbox}
    {$\mathcal{S}_{\mathsf{ledger}}$ (Part 3 - Internal Procedures)}
    {internal-procedures}

    \begin{algorithmWithNumbering}{simulatorCounter}
        \Statex{\underline{$\textsc{SimulateMining}(\party, \tau)$}}
        \Comment{Simulate the mining procedure of \party in the protocol in round $\tau$}

        \State{Execute $\mathsf{FetchInformation}(\party, \sid)$}

        \If{$\mathsf{Update}_{\party, \tau}$}
        \State{Send $(\textsc{clock-update}, \sid_C , \party)$ to \adv if $\mathcal{S}_{\mathsf{ledger}}$ has received such an input in round $\tau$}
        \Else
        \State{Execute $\mathsf{MiningProcedure}(\party, \sid)$ and set $\mathsf{Update}_{\party, \tau} \gets \true$}
        \State{Before the activation goes to \adv, execute $\textsc{ExtendLedgerState}(\tau)$.}
        \EndIf
    \end{algorithmWithNumbering}

    \medskip

    \begin{algorithmWithNumbering}{simulatorCounter}
        \Statex{\underbar{$\textsc{ExtendLedgerState}(\tau)$}}

        \State{Let \txSeqProposal be the longest state (extracted from \chainLocal by $\mathsf{ExtractTransactionSequence}$) among all such states $\txSeqProposal_{U_p}$ where $U_p \in \honestPartySet$}

        \State{Let $\txSeqProposal_\ledger$ denote the sequence reconstructed from \state in \funcFairLedger}

        \oneLineIf{$|\txSeqProposal_\ledger| > |\txSeqProposal^{\lceil \CPLen}|$}{Execute $\textsc{AdjustView}(\state)$}

        \If{$\txSeqProposal_\ledger$ is not a prefix of $\txSeqProposal^{\lceil \CPLen}$}
        \State{\textbf{Abort} simulation: consistency violation.} \Comment{Event \badCP}
        \EndIf

        \State{Define the difference \textsf{diff} to be the block sequence s.t. $\txSeqProposal_\ledger \concat \mathsf{diff} = \txSeqProposal^{\lceil \CPLen}$.}

        \State{Parse $\mathsf{diff} = \mathsf{diff}_1 \concat \ldots \concat  \mathsf{diff}_n$.}

        \For{$j$ \textbf{from} 1 \textbf{to} $n$}
        \State{Map each transaction \tx in this block to its unique transaction ID \txid.}

        \State{Let $\mathsf{list}_j = (\txid_{j,1}, \ldots ,\txid_{j,\ell_j})$ be the corresponding list for this block $\mathsf{diff}_j$}

        \If{coinbase $\txid_{j,1}$ specifies a party honest at block creation time}
        \State{$\mathrm{hFlag} \gets 1$}
        \Else
        \State{$\mathrm{hFlag} \gets 0$}
        \EndIf

        \State{Output $(\textsc{next-block}, \mathsf{list}_j)$ to \funcFairLedger and receive $(\textsc{next-block}, \ok)$ as an immediate answer.}
        \EndFor

        \If{fraction of blocks with $\mathrm{hFlag} = 0$ in the recent \CPLen blocks $> 1 - \mu_{\mathsf{CQ}}$}
        \State{\textbf{Abort} simulation: chain quality violation}
        \Comment{Event \badCQ}

        \ElsIf{state increases less than $k$ blocks during the last $k / \tau_{\mathsf{CG}}$ rounds}
        \State{\textbf{Abort} simulation: chain growth violation}
        \Comment{Event \badCG}

        \ElsIf{honest profiles account for less than half in the last \PBWindowLen blocks}
        \State{\textbf{Abort} simulation: profile block violation}
        \Comment{Event \badProfile}
        \EndIf

        \LineComment{Equivocate honest transactions.}

        \State{Send $(\textsc{read}, \sid)$ to \funcFairLedger and receive \state}

        \For{$\tx \in \state$ \textbf{and} \txid corresponds to a faked transaction \textbf{and} $\mathsf{equivocate}_\tx = \false$}
        \State{Let $r_1, \ldots r_m$ denote the randomenss used to generate $ct_1, \ldots, ct_m$ of \tx (via \txid)}
        
        \For{$i$ \textbf{from} $1$ to m}
        \State{Split $r_i$ into $|\tx|$ pieces and for each piece $r_{i, j}$ send $(\textsc{program}, r_{i,j}, h)$ such that $h \oplus ct_i = \tx[j]$ to \funcGrpoRO}
        \If{programming \funcGrpoRO fails}
        \State{\textbf{Abort} simulation: failure of decryption}
        \Comment{Event \badDec}
        \EndIf
        \EndFor
        
        \State{Set $\mathsf{equivocate}_\tx \gets \true$}
        \EndFor

        \State{Execute $\textsc{AdjustView}(\state)$}
    \end{algorithmWithNumbering}

    \medskip

    \begin{algorithmWithNumbering}{simulatorCounter}
        \Statex{\underline{$\textsc{AdjustView}(\state, \tau)$}}
        \Comment{ Adjust the view of synchronized parties.}

        \State{$\mathsf{pointers} \gets \varepsilon$}

        \For{party $U_p \in \honestPartySet$ of round $\tau$}
        \State{Execute $\mathsf{ReadState}(U_p, \sid)$ and let the chain's decoded state be $\vec{\st}_{U_p}$}
        \EndFor

        \For{each synchronized party $U_p \in (\honestPartySet \backslash \partyset_{DS})$ of round $\tau$}
        \State{Determine the pointers $\pt_{U_p}$ s.t. $\vec{\st}_{U_p}^{\lceil \CPLen} = \state|_{\pt_{U_p}}$}
        \If{such a point does not exist}
        \State{\Return} \Comment{Call on invalid input or event \badCP occurred}
        \EndIf

        \oneLineIf{$\mathsf{Update}_{U_p, \tau}$}{$\mathsf{pointers} \gets \mathsf{pointers} \concat (U_p, \pt_{U_p})$}
        \EndFor

        \State{Output $(\textsc{set-slack}, \mathsf{pointers})$ to \funcFairLedger}

        \LineComment{Adjust the view of desynchronized parties}
        \State{$\mathsf{pointers} \gets \varepsilon$}
        \State{$\mathsf{desyncStates} \gets \varepsilon$}

        \For{each desynchronized party $U_p \in \partyset_{DS}$ of round $\tau$}

        \If{$\mathsf{Update}_{U_p, \tau} = \false$}
        \State{Set $\pt_{U_p}$ to be $|\vec{\st}_{U_p}^{\lceil \CPLen}|$}
        \State{$\mathsf{pointers} \gets \mathsf{pointers} \concat (U_p, \pt_{U_p})$}
        \State{$\mathsf{desyncStates} \gets \mathsf{desyncStates} \concat (U_p, \vec{\st}_{U_p}^{\lceil \CPLen})$}
        \EndIf
        \State{Output $(\textsc{set-slack}, \mathsf{pointers})$ to \funcFairLedger}
        \State{Output $(\textsc{desync-state}, \mathsf{desyncStates})$ to \funcFairLedger}
        \EndFor
    \end{algorithmWithNumbering}

    \medskip

    \begin{algorithmWithNumbering}{simulatorCounter}
        \Statex{\underline{$\textsc{IssueNewTransaction}(\party, \tau, m)$}}
        \Comment{Issue dummy transactions}

        \State{Let \chain denote the simulated blockchain for party \party}

        \State{$h \gets \chainHead{\chain^{\lceil k}}$ and $st \gets$ blockchain state associated with $h$}
        \State{$\mathcal{S}_{\mathsf{pk}} \gets$ all enclave public keys up to block with hash $h$}

        \State{$\mathcal{S}^\ticket_{\mathsf{pk}} \gets f_{\mathsf{select}}(\mathcal{S}_{\mathsf{pk}}, m, \ticket)$ and $m' \gets |\mathcal{S}^\ticket_{\mathsf{pk}}|$}

        \If{$\mathcal{S}^\ticket_{\mathsf{pk}}$ contains only corrupted enclave keys}
        \State{\textbf{Abort} simulation: violation of good block ticket} \Comment{Event \badTicket}
        \EndIf

        \For{$i$ \textbf{from} $1$ \textbf{to} $m'$}
        \Comment{Encrypt an all-zero string}

        \State{$\mathsf{pk}_i \gets \mathcal{S}^\ticket_{\mathsf{pk}}[i], r_i \overset{\$}{\gets} \{0, 1\}^\kappa$}
        \State{$ct_i \gets \PKE.\Enc(\mathsf{pk}_i, 0^m \concat h; r_i)$ and remember $r_i$ for equivocation later}
        \EndFor

        \LineComment{Generate fake NIZK proof, telling an all-zero string is a valid transactino.}
        \State{$\pi \gets \NIZK.\Prove((h, st, (ct_1, \ldots, ct_{m'}), (\mathsf{pk}_1, \ldots, \mathsf{pk}_{m'})),\allowbreak (\tx, (r_1, \ldots, r_{m'})))$}

        \State{\Return $(\pi, ct, \tau)$}
    \end{algorithmWithNumbering}
\end{simulatorbox}


\subsection{Blockchain Security Properties}
\label{subsec:blockchain-security-properties}

In our protocol, a blockchain is built as an intermediate step to agree on transaction timestamp and provide long chains for decrypting transactions.
%
We thus first focus on the good properties on the chain --- namely, common prefix, chain growth and chain quality.

\begin{cccItemize}[noitemsep]
	\item \textbf{Common Prefix ($\mathsf{CP}$); parameterized with $k \in \mathbb{N}^+$.}
	%
	The chains $\chain_1, \chain_2$ possessed by two alert parties at the
	onset of the round $r_1 < r_2$ are such that $\chainPrefix{\chain_1}{k} \preceq \chain_2$, where $\chainPrefix{\chain_1}{k}$ denotes the chain obtained by removing the last $k$ blocks from $\chain_1$, and $\preceq$ denotes the prefix relation.

	\item \textbf{Chain Growth ($\mathsf{CG}$); parameterized with $\tau \in (0, 1]$ and $s \in \mathbb{N}$.}
	% 
	Consider a chain \chain possessed by an alert party at the onset of a round $r$. Let $r_1$ and $r_2$ be two previous rounds for which $r_1 + s \le r_2 \le r$.
	%
	Then $|\chain[r_1 : r_2]| \ge \tau \cdot s$ where $\tau$ is the speed coefficient.

	\item \textbf{Chain Quality ($\mathsf{CQ}$); parameterized with $\mu \in (0, 1]$ and $s \in \mathbb{N}$.}
	%
	Consider any portion of length at least $s$ of the chain possessed by an alert party at the onset of a round; the ratio of blocks originating from alert parties is at least $\mu$. We call $\mu$ the chain quality coefficient.
\end{cccItemize}

Note that our blockchain framework can be viewed as a ``superset'' of the Bitcoin backbone protocol, where they share the same chain structure (recall that all our revisions are regarding letting the chain accept new types of ``profile blocks'' and the transaction encryption scheme is blockchain-agnostic).
%
Hence, following the similar arguments in~\cite{EC:GarKiaLeo15,EC:PasSeeash17,C:BMTZ17}, the following~\cref{lemma:blockchain-properties} concluding that these properties hold throughout the entire execution except with negligible probability is immediate.
%
Note that to simplify the presentation, we omit the detailed parametrization of the protocol but only focus on the achieved properties.
%
Specifically, common prefix is achieved by pruning last $\CPLen = \polylog(\kappa)$ blocks, and the chain quality parameter $\mu > 0$, indicating that there is at least one honest block for every \CPLen consecutive blocks.

\begin{lemma} \label{lemma:blockchain-properties}
	There exist protocol parametrizations such that the following properties hold throughout the an execution of \protocFairLedger for $L = \poly(\kappa)$ rounds: (i) common prefix property with parameter $\CPLen = \Theta(\polylog(\kappa))$ (ii) chain growth property with parameter $\tau_{\mathsf{CG}}$ and $s_{\mathsf{CG}}$; and (iii) chain quality with parameter $\mu > 0$ and $s = \CPLen$, except with probability negligibly small in the security parameter $\kappa$.
\end{lemma}

We next focus on the properties regarding profile blocks.
%
We show that honest parties can always produce profile blocks that account for the majority, for any $\PBWindowLen = \Theta(\CPLen)$ consecutive blocks on the blockchain.
%
We provide a sketched proof using only the common prefix, chain growth and chain quality parameter; a more refined proof and parametrization can be found in, e.g., \cite{PODC:PasShi17,EC:KiaLeoShe24}.

\begin{lemma}
	[Majority of honest profile blocks]
	\label{lemma:good-profile-blocks}
	If the properties as in~\cref{lemma:blockchain-properties} are not violated during the execution, then in an execution of \protocFairLedger over a lifetime of $L = \poly(\kappa)$ rounds, for any segmentation of the blockchain of $\PBWindowLen = \Theta(\CPLen)$ consecutive blocks, the majority of the profile blocks included are produced by honest parties.
\end{lemma}


\begin{proof}[Proof (Sketch)]
	Consider any consecutive $K$ blocks $\block_i, \ldots \block_{i + \PBWindowLen -1}$.
	%
	Due to chain quality, there is at least one honest block $\block_h$ such that  $i \le h < i + \CPLen -1 $ and $\block_{h'}$ such that $i + \PBWindowLen - \CPLen < h' \le i + \PBWindowLen - 1$.
	%
	It suffices to show that the profile blocks produced by honest parties in the time interval between $\block_h$ and $\block_{h'}$ is larger than those produced by the corrupted parties in an interval between the time of $t$ and $t'$ where $t = \mathsf{timestamp}(\block_i) - R$ and $t' = \mathsf{timestamp}(\block_{i + \PBWindowLen - 1}) + \CPLen$.
	%
	By setting $R = 3\CPLen$ and $\PBWindowLen = \Theta(\CPLen)$ (the constants depends on the advantage of honest computational power compared with the corrupted parties, and the chain growth parameter $\tau_{\mathsf{CG}}$) we conclude the proof.
\end{proof}

Given that for any sliding window of \PBWindowLen blocks the majority of the profile blocks included are produced by honest parties, we consider a transaction \tx enter the system at time $t$ (when \tx is issued by corrupted parties, let $t$ denote the earliest time such that \tx is learnt by at least one honest party).
%
Every honest maintainer will receive \tx at a time $t'$ such that $t \le t' < t + \delay$, hence majority of the profile blocks report time in this interval --- i.e., in the final ledger \ledger, the position of \tx will be later than any transaction that enters the system at time before $t - \delay$; meantime, \tx is positioned at a place earlier than those transactions entering the system at time later than $t + \delay$.

\subsection{Composable Guarantees}
\label{subsec:composable-guarantees}

\cref{subsec:blockchain-security-properties} shows that when the simulator jointly builds the blockchain with the black-box adversary, bad events regarding the violation of good blockchain properties, namely \badCP, \badCQ, \badCG and \badProfile, happens with negligible probability.
%
We next prove that the simulator \simulator can simulate the transaction encryption mechanism well, by showing that all bad events \badTicket, \badNIZK and \badDec happen with negligible probability.
%
Our proof makes use of the following large convergence bound.

\begin{theorem}
	[Chernoff bounds]
	\label{thm:chernoff-bounds}

	Suppose $\{X_i: i \in [n]\}$ are mutually independent Boolean random variables, with $\Pr[X_i = 1] = p$, for all $i \in [n]$. Let $X =\sum^n_{i = 1} X_i$ and $\mu = pn$.
	%
	Then, for any $\delta \in (0, 1]$, it holds that
	%
	\[
		\Pr [X \le (1 - \delta) \mu] \le e^{- \delta^2 \mu / 2}
		~~\text{and}~~
		\Pr [X \ge (1 + \delta) \mu] \le e^{- \delta^2 \mu / 3}.
	\]
	%
	Also, for all $t > 0$,
	%
	\[ \Pr[X \ge \mu + t] \le e^{-2 t^2 n}. \]
\end{theorem}

First, we consider \badTicket which implies that the adversary \adv can send an encrypted transaction such that all public keys used in the encryption are from enclaves controlled by the corrupted parties thus \adv can decided whether to open the transaction or not.

\begin{lemma}
	[Good block tickets]
	\label{lemma:good-block-tickets}
	Consider an execution of \protocFairLedger over a lifetime of $L = \poly(\kappa)$ rounds.
	%
	The probability such that there exists a block ticket $(h, \ticket)$ such that $h$ equals the hash of a block in the settled part of the blockchain held by an honest party (at any time during the execution) is negligible with respect to the security parameter $\kappa$.
\end{lemma}

\begin{proof}
	Recall that the mining target $T$ is appropriately parameterized such that the block generation rate $f < 1$ is a small constant, by applying Chernoff bound (\cref{thm:chernoff-bounds}), the number of blocks generated in $L = \poly(\kappa)$ rounds is bounded by $(1 + \epsilon)f \cdot L$ except with probability $\exp(- \epsilon^2 f L / 3) = \exp(-\Omega(\poly(\kappa)))$ which is negligibly small with respect to $\kappa$.
	%
	I.e., consider $n$ enclaves, at most $n \cdot \poly(\kappa)$ block tickets will be associated with hash of a block in the settled blockchain.

	Let $p \in (0 ,1]$ denote the fraction of honest enclave public keys.
	%
	We first consider a variant of~\cref{program:block-ticket} where for each hash $h$, the enclave returns a randomly selected value (and records this value for $h$ for later queries), which applies a random subset selection on the set of enclave public-key set $\mathcal{S}_\pk$ for each $h$.
	%
	We show that by randomly select a subset of size $m = \polylog (\kappa)$ in $\mathcal{S}_\pk$ for $n \cdot \poly(\kappa)$ times, the probability such that event $E$ --- there exists at least one selected subset such that all keys are from the enclaves controlled by corrupted corrupted parties --- is negligible.
	%
	For each random subset selection, the probability that no honest key is selected is bounded by $(1 - p)^m < \exp(-\Omega(\polylog\kappa))$, by selecting $n \cdot \poly(\kappa)$ we get $\Pr[E] = 1 - (1 - \exp(-\Omega(\polylog\kappa)))^{n \cdot \poly(\kappa)} = \exp(-\Omega(\polylog(\kappa)) + \ln n)$ which is negligible in $\kappa$.

	Now it suffices to show that by replacing the random number with $\fPRF(\mathsf{key}, h)$, for each ticket there is still at least one key controlled by honest party get selected.
	%
	This can be proved by a reduction to the indistinguishability experiment of the underlying pseudorandom function \fPRF which we omit here (note that while each enclave possesses her own key, since the total number of the enclaves are polynomially bounded this does not hurt the argument).
\end{proof}

Then we consider \badNIZK which implies that the simulator \simulator cannot extract the encrypted transactions issued by the corrupted parties thus the simulation fails (note that it is not hard to see that the event such that honest parties fail to create a proof, or the corrupted parties generate a proof for a relation not in~\cref{eq:nizk-relation} is negligible).

\begin{lemma}
	[Good NIZK]
	\label{lemma:good-nizk}
	The probability that the simulator fails to extract the witness from a NIZK proof created by the black-box adversary is negligibly small in the security parameter $\kappa$.
\end{lemma}

\begin{proof}
	We prove this by a reduction to the security game of the special simulation-soundness (\cref{game:nim-sss} in~\cref{def:non-interactive-special-simulation-soundness}), and show that if the simulator fails to extract, then the game returns \texttt{fail} with non-negligible probability, contradicting the fact that \NIZK is non-interactive special simulation-soundness.

	The reduction in general works as follows.
	%
	When the simulator \simulator is to generate a NIZK proof for dummy honest parties, \simulator forwards the query $(\mathtt{Prove}, x, w)$ as that in~\cref{game:nim-sss}; when receiving a proof $(x, \pi)$ from the black-box adversary, \simulator forwards $(\mathtt{Challenge}, x, \pi)$ and extracts the witness.
	%
	If this extraction fails and simulator aborts, then in~\cref{game:nim-sss} it must be the case that after running the $\mathsf{Extract}$ algorithm and get witness $w$ it holds that $R(x, w) = 0$.
	%
	I.e., if the simulator aborts for the failed extraction with non-negligible probability, then in~\cref{game:nim-sss} it also returns \texttt{Fail} with non-negligible probability.
\end{proof}

Recall that in~\cref{lemma:pke-indistinguishability}, the black-box adversary \adv cannot distinguish if a plaintext of \PKE is decrypted by \Dec or $\Dec^*$, finally we consider the event \badDec which implies that the simulator fails to equivocate a faked ciphertext to its corresponding honest transaction.

\begin{lemma}
	[Good decryption]
	\label{lemma:good-decryption}
	The probability such that when \adv queries \funcEnclave a faked ciphertext $ct$ of an all-zero string corresponding to transaction \tx and \funcEnclave returns a response other than $(\textsc{tx-dec}, (\tx, ct))$ is negligible with respect to the security parameter $\kappa$.
\end{lemma}

\begin{proof}
	The event of \badDec happens, because either \simulator has no time to program the random oracle, or the programming fails.

	We first consider the case that \simulator has no time to program.
	%
	Note that since the decryption parameter is set as $\Lambda > 4\CPLen + \PBWindowLen$, for an honest transaction \tx that associates with a block with height $\ell$, there will exist an honest block with height at most $\ell + 2\CPLen$ (due to chain quality) such that it contains the profile blocks with \tx.
	%
	I.e., when the blockchain has progressed to length $\ell + 4\CPLen + \PBWindowLen$, the timestamp of \tx get settled, and the same holds for all transactions that enters the system before \tx.
	%
	Hence \simulator can propose a block containing \tx and learn its plaintext before the chain grows to length $\ell + \Lambda > \ell + 4\CPLen + \PBWindowLen$; i.e., when the simulator program all ciphertext for \tx the \textsc{tx-dec} interface will never try to use $\Dec^\ast$ to decrypt $ct$ of \tx.

	Regarding the event that programming \funcGrpoRO fails, note that this only happens when the point $x$ to program has already been queried before.
	%
	Note that for any point $x$ to program, there exists a (piece of) ciphertext $ct$ with respect to $\mathsf{pk} = (\mathsf{f}, \mathsf{b})$, such that $ct = \mathsf{f}(x)$.
	%
	If the event that any programming on point $x$ fails, then it implies $\Pr_{x \in\{0 ,1\}^\kappa}[\adv(f(x)) \in \mathsf{f}\inv(\mathsf{f}(x))]$ is non-negligible, contradicting the fact that $\mathsf{f}$ is a secure one-way function.
\end{proof}

We conclude that our protocol \protocFairLedger, when appropriately parameterized, securely realizes \funcFairLedger in an $(\funcClock,\allowbreak \wrapper{\funcGRO},\allowbreak \funcDiffuse,\allowbreak \funcEnclave,\allowbreak \funcGrpoRO)$-hybrid environment.

\begin{theorem}
	Let \delay denote the network delay, $\tau_{\mathsf{CG}}$ the chain-growth coefficient, \CPLen the common prefix parameter, \PBWindowLen the length of profile block window and $\PKE, \Lambda$ the parameter of \cref{program:tx-decryption} where \PKE is as specified in~\cref{algorithm:pke}.
	%
	Assuming honest majority in terms of computational power and honest parties controlling a constant fraction of enclave public keys, there exists protocol parametrization ($\CPLen = \Theta(\delay \log^2 \kappa)$, $\PBWindowLen = \Theta(k)$ and $\Lambda = \PBWindowLen + 5\CPLen$, $h / n = \Omega(\log^2 \kappa)$) such that Protocol \protocFairLedger UC-realizes \funcFairLedger parameterized with \delay-\textsf{ApproxSenderOrder}, with $\mathtt{windowSize} = k$, $\mathtt{Delay} = 2\delay$ and $\mathtt{waitTime} = (4\CPLen + \PBWindowLen) / \tau_{\mathsf{CG}}$, in the $(\funcClock,\allowbreak \wrapper{\funcGRO},\allowbreak \funcDiffuse,\allowbreak \funcEnclave,\allowbreak \funcGrpoRO)$-hybrid environment.
\end{theorem}

\begin{proof}[Proof (Sketch)]
	We prove the theorem by providing a simulator $\mathcal{S}_{\mathsf{ledger}}$ in the ideal world such that the protocol execution in the  $(\funcClock,\allowbreak \wrapper{\funcGRO},\allowbreak \funcDiffuse,\allowbreak \funcEnclave,\allowbreak \funcGrpoRO)$-hybrid world is indistinguishable from the ideal-world execution with the ledger functionality and the simulator.
	%
	This simulation can be done perfectly, as the only events that prevent a successful simulation are those defined by \badCP, \badCQ, \badCG, \badProfile, \badNIZK and \badDec, which we have been proving in the previous \cref{lemma:blockchain-properties,lemma:good-profile-blocks,lemma:good-block-tickets,lemma:good-nizk,lemma:good-decryption} that happens with negligible probability.
\end{proof}

